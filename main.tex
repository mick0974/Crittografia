\documentclass[a4paper,11pt]{book}

\usepackage[utf8]{inputenc}
\usepackage[italian]{babel}
\usepackage[T1]{fontenc}
\usepackage[letterpaper,top=2cm,bottom=2cm,left=3cm,right=3cm,marginparwidth=1.75cm]{geometry}

% Useful packages
\usepackage{amsmath}
\usepackage{graphicx}
\usepackage[colorlinks=true, allcolors=blue]{hyperref}
\usepackage{listings}
\usepackage{amsthm}
\usepackage{amssymb}

\usepackage{algorithm}
\usepackage{algpseudocode}

\theoremstyle{plain}
\newtheorem{theorem}{Teorema}[chapter]
\theoremstyle{plain}
\newtheorem{lemma}{Lemma}[chapter]
\theoremstyle{definition}
\newtheorem{definition}{Definizione}[chapter]

\usepackage{float}

\title{Crittografia}

\begin{document}
\maketitle

\thispagestyle{empty}

\tableofcontents

\thispagestyle{empty}
	
\cleardoublepage %%%% Add this one here!
\pagenumbering{arabic}

\input{chapter1}
\input{chapter2 - crittogradia in cascata}
\input{chapter3 - cenni alla teoria dei numeri}
\input{chapter4 - crittografia a chiave pubblica}
\input{chapter5 - crittografia a blocchi}
\input{chapter6 - codifica di un singolo bit}
\input{chapter7 - codifica di sequenze di bit}
\input{chapter8 - lancio della moneta e hard core predicate}
\input{chapter9 - generatori pseudocasuali}
\chapter{Funzioni pseudocasuali}
\label{chapter9}

\chapter{Protocolli vari}
\label{chapter1}

 










\end{document}